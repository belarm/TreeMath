\documentclass{article}
\usepackage{amsfonts}
\begin{document}

\input names.tex
We shall define boolean strings $\boolstring$ as:
\[\boolspace = \{False, True\}\]
\[B_1 = \{b_0 : b_0 \in \boolspace\}\]
\[B_2 = \{b_0 b_1 : b_0 \in \boolspace \wedge b_1 \in \boolspace\}\]
\[...\]
\[B_n = \{b_0 b_1 ... b_{n-1} : b_0 ... b_{n-1} \in \boolspace\}\]
\[\boolstring = \{B_1, B_2, ..., B_\infty\}\]

Pretty sure I don't need $\boolstring$, but it's pretty. I think $B_n$ is what I really want.

Thus, a boolean string is a non-terminating, positionally-ordered
set of $\{False, True\}$ taking the form \textit{eg} $00100011000....$

While strings do not terminate, they may have an infinite number of trailing $False$ entries.
Such strings can be simply represented in a computer's memory, but the full expansion of $b \in \boolstring$ cannot.

This definition removes much of the utility of boolean strings in computers, but does allow them to fully represent $\mathbb{R}$, for example.

We shall attempt to show that, through techniques such as Dedekind cuts, one can embed all spaces within $\boolstring$.
\end{document}
